\documentclass[journal,12pt,twocolumn]{IEEEtran}

\usepackage{setspace}
\usepackage{gensymb}

\singlespacing


\usepackage[cmex10]{amsmath}

\usepackage{amsthm}

\usepackage{mathrsfs}
\usepackage{txfonts}
\usepackage{stfloats}
\usepackage{bm}
\usepackage{cite}
\usepackage{cases}
\usepackage{subfig}

\usepackage{longtable}
\usepackage{multirow}

\usepackage{enumitem}
\usepackage{mathtools}
\usepackage{steinmetz}
\usepackage{tikz}
\usepackage{circuitikz}
\usepackage{verbatim}
\usepackage{tfrupee}
\usepackage[breaklinks=true]{hyperref}
\usepackage{graphicx}
\usepackage{tkz-euclide}
\usepackage{float}

\usetikzlibrary{calc,math}
\usepackage{listings}
    \usepackage{color}                                            %%
    \usepackage{array}                                            %%
    \usepackage{longtable}                                        %%
    \usepackage{calc}                                             %%
    \usepackage{multirow}                                         %%
    \usepackage{hhline}                                           %%
    \usepackage{ifthen}                                           %%
    \usepackage{lscape}     
\usepackage{multicol}
\usepackage{chngcntr}

\DeclareMathOperator*{\Res}{Res}

\renewcommand\thesection{\arabic{section}}
\renewcommand\thesubsection{\thesection.\arabic{subsection}}
\renewcommand\thesubsubsection{\thesubsection.\arabic{subsubsection}}

\renewcommand\thesectiondis{\arabic{section}}
\renewcommand\thesubsectiondis{\thesectiondis.\arabic{subsection}}
\renewcommand\thesubsubsectiondis{\thesubsectiondis.\arabic{subsubsection}}


\hyphenation{op-tical net-works semi-conduc-tor}
\def\inputGnumericTable{}                                 %%

\lstset{
%language=C,
frame=single, 
breaklines=true,
columns=fullflexible
}
\begin{document}


\newtheorem{theorem}{Theorem}[section]
\newtheorem{problem}{Problem}
\newtheorem{proposition}{Proposition}[section]
\newtheorem{lemma}{Lemma}[section]
\newtheorem{corollary}[theorem]{Corollary}
\newtheorem{example}{Example}[section]
\newtheorem{definition}[problem]{Definition}

\newcommand{\BEQA}{\begin{eqnarray}}
\newcommand{\EEQA}{\end{eqnarray}}
\newcommand{\define}{\stackrel{\triangle}{=}}
\bibliographystyle{IEEEtran}
\providecommand{\mbf}{\mathbf}
\providecommand{\pr}[1]{\ensuremath{\Pr\left(#1\right)}}
\providecommand{\qfunc}[1]{\ensuremath{Q\left(#1\right)}}
\providecommand{\sbrak}[1]{\ensuremath{{}\left[#1\right]}}
\providecommand{\lsbrak}[1]{\ensuremath{{}\left[#1\right.}}
\providecommand{\rsbrak}[1]{\ensuremath{{}\left.#1\right]}}
\providecommand{\brak}[1]{\ensuremath{\left(#1\right)}}
\providecommand{\lbrak}[1]{\ensuremath{\left(#1\right.}}
\providecommand{\rbrak}[1]{\ensuremath{\left.#1\right)}}
\providecommand{\cbrak}[1]{\ensuremath{\left\{#1\right\}}}
\providecommand{\lcbrak}[1]{\ensuremath{\left\{#1\right.}}
\providecommand{\rcbrak}[1]{\ensuremath{\left.#1\right\}}}
\theoremstyle{remark}
\newtheorem{rem}{Remark}
\newcommand{\sgn}{\mathop{\mathrm{sgn}}}
\providecommand{\abs}[1]{\left\vert#1\right\vert}
\providecommand{\res}[1]{\Res\displaylimits_{#1}} 
\providecommand{\norm}[1]{\left\lVert#1\right\rVert}
%\providecommand{\norm}[1]{\lVert#1\rVert}
\providecommand{\mtx}[1]{\mathbf{#1}}
\providecommand{\mean}[1]{E\left[ #1 \right]}
\providecommand{\fourier}{\overset{\mathcal{F}}{ \rightleftharpoons}}
%\providecommand{\hilbert}{\overset{\mathcal{H}}{ \rightleftharpoons}}
\providecommand{\system}{\overset{\mathcal{H}}{ \longleftrightarrow}}
	%\newcommand{\solution}[2]{\textbf{Solution:}{#1}}
\newcommand{\solution}{\noindent \textbf{Solution: }}
\newcommand{\cosec}{\,\text{cosec}\,}
\providecommand{\dec}[2]{\ensuremath{\overset{#1}{\underset{#2}{\gtrless}}}}
\newcommand{\myvec}[1]{\ensuremath{\begin{pmatrix}#1\end{pmatrix}}}
\newcommand{\mydet}[1]{\ensuremath{\begin{vmatrix}#1\end{vmatrix}}}
\numberwithin{equation}{subsection}
\makeatletter
\@addtoreset{figure}{problem}
\makeatother
\let\StandardTheFigure\thefigure
\let\vec\mathbf
\renewcommand{\thefigure}{\theproblem}
\def\putbox#1#2#3{\makebox[0in][l]{\makebox[#1][l]{}\raisebox{\baselineskip}[0in][0in]{\raisebox{#2}[0in][0in]{#3}}}}
     \def\rightbox#1{\makebox[0in][r]{#1}}
     \def\centbox#1{\makebox[0in]{#1}}
     \def\topbox#1{\raisebox{-\baselineskip}[0in][0in]{#1}}
     \def\midbox#1{\raisebox{-0.5\baselineskip}[0in][0in]{#1}}
\vspace{3cm}
\title{Challenge Problem 1}
\author{K.A. Raja Babu}
\maketitle
\newpage
\bigskip
\renewcommand{\thefigure}{\theenumi}
\renewcommand{\thetable}{\theenumi}
Download latex-tikz code from 
%
\begin{lstlisting}
https://github.com/ka-raja-babu/Matrix-Theory/tree/main/ChallengeProblem1
\end{lstlisting}
%
\section{Challenge Question 1}
Show that the matrix $(t\vec{I}-\vec{n}\vec{n}^T)$ in the given document is a rank 1 matrix for a parabola.
%
\section{Solution}
Let
\begin{align}
    \vec{V} = (t\vec{I}-\vec{n}\vec{n}^T) 
\end{align}
where,
\begin{align}
    t &= \frac{\norm{\vec{n}}^2}{e^2}
\end{align}

Let rank of matrix be represented by $\rho$ and trace of matrix be represented by $tr$

Now,
\begin{align}
    tr(\vec{V}) &= tr(t\vec{I}-\vec{n}\vec{n}^T)
    \\
    \implies tr(\vec{V}) &= tr(t\vec{I}) - tr(\vec{n}\vec{n}^T)
    \\
    \implies tr(\vec{V}) &= 2t - \norm{\vec{n}}^2
    \\
    \implies tr(\vec{V}) &= \frac{\norm{\vec{n}}^2(2-e^2)}{e^2} \label{trace}
\end{align}

Now,
\begin{align}
    \vec{V}^T &= (t\vec{I}-\vec{n}\vec{n}^T)^T
    \\
    \implies \vec{V}^T &= (t\vec{I})^T-(\vec{n}\vec{n}^T)^T
    \\
    \implies \vec{V}^T &= t(\vec{I}^T) - ((\vec{n}^T)^T\vec{n}^T)
    \\
    \implies \vec{V}^T &= (t\vec{I}-\vec{n}\vec{n}^T)
    \\
    \implies \vec{V}^T &= \vec{V}    \label{symm}
\end{align}
So,using \eqref{symm},$\vec{V}$ is a symmetric matrix .

\begin{theorem}
\label{theorem1}
Rank of a 2x2 symmetric matrix $\vec{A}$ is given by
\begin{align}
    \rho(\vec{A}) &= \frac{(tr(\vec{A}))^2}{tr(\vec{A}^2)}
\end{align}
\end{theorem}

\begin{theorem}
For a 2x2 matrix $\vec{A}$
\label{theorem2}
\begin{align}
    tr(\vec{A}^2) &= (tr(\vec{A}))^2 - 2\abs{\vec{A}}
\end{align}
\end{theorem}

Using theorem \ref{theorem2}
\begin{align}
    tr(\vec{V}^2) &= (\frac{\norm{\vec{n}}^2(2-e^2)}{e^2})^2 - 2\abs{\vec{V}}
\end{align}

Using theorem \ref{theorem1}
\begin{align}
    \rho(\vec{V}) &= \frac{(\frac{\norm{\vec{n}}^2(2-e^2)}{e^2})^2}{(\frac{\norm{\vec{n}}^2(2-e^2)}{e^2})^2 - 2\abs{\vec{V}}}
\end{align}

For a parabola where $e$=1 and $\abs{\vec{V}}$=0,
\begin{align}
    \rho(\vec{V}) &= \frac{\norm{\vec{n}}^4}{\norm{\vec{n}}^4}
    \\
    \implies \rho(\vec{V}) &= 1
\end{align}
\end{document}

